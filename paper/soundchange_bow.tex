%
% File emnlp2016.tex
%

\documentclass[11pt,letterpaper]{article}
\usepackage{emnlp2016}
\usepackage{times}
\usepackage{latexsym}

% Uncomment this line for the final submission:
%\emnlpfinalcopy

%  Enter the EMNLP Paper ID here:
\def\emnlppaperid{***}

% To expand the titlebox for more authors, uncomment
% below and set accordingly.
% \addtolength\titlebox{.5in}    

\newcommand\BibTeX{B{\sc ib}\TeX}


\title{Universal Sound change for Bag-of-Words Models}

% Author information can be set in various styles:
% For several authors from the same institution:
% \author{Author 1 \and ... \and Author n \\
%         Address line \\ ... \\ Address line}
% if the names do not fit well on one line use
%         Author 1 \\ {\bf Author 2} \\ ... \\ {\bf Author n} \\
% For authors from different institutions:
% \author{Author 1 \\ Address line \\  ... \\ Address line
%         \And  ... \And
%         Author n \\ Address line \\ ... \\ Address line}
% To start a seperate ``row'' of authors use \AND, as in
% \author{Author 1 \\ Address line \\  ... \\ Address line
%         \AND
%         Author 2 \\ Address line \\ ... \\ Address line \And
%         Author 3 \\ Address line \\ ... \\ Address line}
% If the title and author information does not fit in the area allocated,
% place \setlength\titlebox{<new height>} right after
% at the top, where <new height> can be something larger than 2.25in
\author{Siddharth Patwardhan \and Daniele Pighin\\
  {\tt publication@emnlp2016.net}}

\date{}

\begin{document}

\maketitle

\begin{abstract}
  This document contains instructions for preparing EMNLP 2016 submissions
  and camera-ready manuscripts.  The document itself conforms to its own
  specifications, and is therefore an example of what your manuscript
  should look like.  Papers are required to conform to all the directions
  reported in this document. By using the provided \LaTeX\ and
  \BibTeX\ styles ({\small\tt emnlp2016.sty}, {\small\tt emnlp2016.bst}),
  the required formatting will be enabled by default.
\end{abstract}


\section{Introduction}

Bag-of-words (BOW) methods are widely used in the industry and academic research in various fields such as Topic modeling and document clustering.
BOW models disregards the order in which words appear in, and consider only the frequencies of words (and generally n-grams).
There are several common practices prior to applying a BOW model, such as removing stop-words using a predefined list, filtering frequent words, stemming and lemming.
The majority of these methods require a prior knowledge about the target language in which the document is written in.
While in fact, language switching, and word borrowing between languages make the process of maintaining a dictionary a complex task.
In this paper we suggest an alternative approach to stemming that takes advantage of universal phonological changes that occur in all human languages.
We apply Monte-Carlo methods on individual words, and imitate the systematic sound shifts that causes the languages and dialects to diverge.
Every word is mapped to a set of possible variations, with the probability the change occurred. We show that cognate words usually converge to the same variations, and thus allow the BOW to better capture the meaning behind the words. 
We describe a method of applying this approach in a scalable manner using the Map-Reduce pattern, while focusing on large quantities of short documents (such as emails, tweets, etc).

\section{Related work}

\section{Model}

\section{Results}

\section{Conclusions}


\section*{Acknowledgments}

Do not number the acknowledgment section.

\bibliography{emnlp2016}
\bibliographystyle{emnlp2016}

\end{document}
